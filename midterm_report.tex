\documentclass[12pt]{article}
\usepackage[utf8]{inputenc}
\usepackage{graphicx}
\usepackage{fancyhdr}
\usepackage{gensymb}
\usepackage{multicol}
\usepackage{color,soul}
\usepackage[colorlinks=true, linkcolor=black, urlcolor=cyan]{hyperref}
\usepackage[margin=1.0in, bmargin=1.2in]{geometry}
\usepackage{wrapfig}
\usepackage{mathtools}
\usepackage{caption}
\usepackage{verbatim}
\usepackage{textcomp}
\usepackage{leftidx}
\setlength{\parindent}{4em}
\setlength{\parskip}{0.1em}
\renewcommand{\baselinestretch}{1}
\usepackage[english]{babel}
\usepackage[nottoc]{tocbibind}
\begin{document}
\rmfamily % Changes font style to times new roman

\begin{titlepage}
\begin{center}
\includegraphics[height=3cm]{CornellLogo.png}\\[0.2cm]
\textsl{\LARGE ORIE 4741 }\\[0.5cm]
{\huge Fall 2017}\\[0.2cm]
\centering
\rule{\linewidth}{0.5mm}\\[0.4cm]
{\LARGE Project Midterm Report \\[0.4cm] Airbnb Price and Occupancy Rate Prediction in Seattle}
\rule{\linewidth}{0.5mm}\\[0.4cm]
\textsl{\Large  Kerou Gao (kg486)\\Kartikay Gupta (kg477)\\}
\vspace{5mm}
\textsl{\Large Cornell University, Ithaca, NY}

\textsl{\Large{October 27, 2017}}\\[9cm]

\end{center}

\noindent \textbf{Acknowledgement}: We would like to thank \textbf{Mr. Jialin Liu}, a PhD student in the Electrical and Computer Engineering Department, for providing valuable inputs in this project.

\end{titlepage}

% set up header and footer
\pagestyle{fancy}
\fancyhf{}
\setlength{\headheight}{50pt}
\renewcommand{\headrulewidth}{0.4pt}
\renewcommand{\footrulewidth}{0.4pt}
\lhead{\includegraphics[height=15mm]{cornell_logo_header.png}}
\rfoot{Fall 2017}
\cfoot{\thepage}



\section*{Introduction}
After considering our primary objective that using review score prediction to reveal important features of Airbnb's listing and improve customer satisfaction, we find that the best way for owners to improve performances is reading reviews, which makes a complicated machine learning model unnecessary. Thus, we change our objective to price and occupancy rate prediction, which can be used for:
\begin{itemize}
\item Guiding new owners to set prices for their properties.
\item Optimizing revenues from both Owners' and Airbnb's perspective. Our models can be used to reveal relationship between price, occupancy rate and other features as well as the trade-off between price and occupancy rate. Since revenue = price * occupancy rate, it is possible to optimize revenue by adjusting price and some characteristics of houses. 
\end{itemize}


\section*{Data Set Description and Feature Selection}
The data set that we use for training our models has been obtained from ``Inside Airbnb" and contains information about 3818 Airbnb listings in Seattle, Washington. There are in total 93 features that are provided in the data set. These features give us a plethora of information ranging from:

\begin{itemize}
\item Listing Location - Coordinates, Neighbourhood, Street
\item Listing Characteristics - Room type, Amenities, Number of Bedrooms, Annual Availability, Number of People that can be accommodated, Transit proximity information etc.
\item Listing Reviews - Statements by guests, number of reviews, review scores etc.
\item In addition to all this, some columns contain details such as host id, scrape id etc. These are obviously useless and will not be used for the development of the predictive model.
\end{itemize}

\noindent Thus the data set that we use has a a mix of discrete, continuous and nominal values. There are some ordinal values as well such as the cancellation policy and response fequency of the Airbnb host.


\noindent For our preliminary analysis, we extracted all the continuous and discrete values from our data set and created a new data frame from these values. The new data set consisted of 30 features in total excluding the price. Out of these features, we used data visualization tools such as bubble plots (Figure 3) and strip plots (Figure 2) to select 7 features that displayed correlation with the listing price. In addition, we also added an offset to get a better fit to our data.

\noindent The feature space that we used consisted of:
\begin{itemize}
\item accommodates: number of persons that can be accommodated in a listing
\item bathrooms: number of bathrooms
\item bedrooms: number of bedrooms
\item beds: number of beds
\item guests included: number of guests included in a listing price
\item latitude \& longitude: coordinates of the listing
\end{itemize}

\noindent 

\begin{center}
\includegraphics[width=8cm]{Price-hist.png}
\includegraphics[width=8cm]{price-bedrooms.png}
\end{center}
\\\textbf{Figure 1}: Price frequency distribution diagram \,\,\,\,\,\,\,
   \textbf{Figure 2}:Correlogram Price-bedrooms


\begin{center}
\includegraphics[width=14cm]{geo_bubble.png}
\\\textbf{Figure 3}: Bubble plot depicting listing price as a function of location
\end{center}


\noindent We find that most features such as number of bedrooms, accommodates etc have positive correlations with price. In terms of geographical location, prices are likely to be highers in certain areas. 

\noindent We also observed a key anomaly in our data set namely that the area of the listing was missing. This certainly would have played a major role in predicting the price of the listing as one would expect a higher price for a larger property.

\section*{Preliminary Analysis and Results}
\noindent After we dropped all the samples with missing values in one or more columns, there were 3796 samples left. This data was then split with 80\% data in the training set and the remaining 20\% in the test set.Our preliminary analysis consisted of fitting some regression models using Scikit Learn in Python to a reduced data set.

\noindent We started by fitting some models such as linear, lasso and ridge regression models and random forest ensemble learning method. We used the root mean squared error (RMSE) as a metric to judge the performance of these learning methods and the table below gives the results obtained for this metric:

\begin{center}
\textbf{Table 1}: Performance of Different Supervised Learning Methods \\[0.3cm] 
\begin{tabular}{ |c|c|c| } 
 \hline
 \multicolumn{3}{|c|}{\textbf{Supervised Learning Method}}\\
 
 \hline 

  & \textbf{Training RMSE (\$)}  & \textbf{Test RMSE (\$)} \\ [0.1cm]
 \hline
  Linear Regression & 64.26 & 66.60 \\ 
 \hline
 Ridge Regression &64.28   &66.60   \\ 
 \hline
 Lasso Regression   &  64.59 & 66.90 \\
 \hline
 Random Forest &52.16  &72.04  \\
 \hline

\end{tabular}

\end{center}


\section*{Further Work}
\noindent After running preliminary analyses, we can see that both
training set errors and test set errors are high,which means that our model is under-fitting. Additionally, we also observe that the errors that we obtain from fitting different regression models are similar. This seems to indicate that there are heavy correlations among our features. Thus the solution to this is to increase the size and diversity of our feature space or use a more complex model. Our plan is listed below:

\noindent (1) Go though non-numerical features and select more into the data set in the sequence of binary value, ordinal values, nominal values and text(description). 

\noindent (2) Try different feature engineering methods to increase complexity
of our models and also use cross validation for hyper parameter tuning. Our ultimate goal is to develop a predictive model that generalizes well on unseen data. 

\noindent (3) Once we have an accurate predictive model for predicting the listing price, we will develop a model for predicting the number of reviews per month for a listing which will act as a proxy for measuring the occupancy rate.



\end{document}